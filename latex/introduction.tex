\chapter{Introduction} \label{Introduction}

Manually processing text-based data is a common task in everyday life.
In hospitals, writing is a major time sink for doctors every day.
As all of this is done in a hurry, it increases the probability for human error when entering critical data to a system.
Thus, using computers to help in the process of data entering and validation provides meaningful value.

Automatic text processing has progressed immensely in the last decade due to major strides in machine learning and especially deep learning -based techniques.
Nowadays, huge deep neural networks dominate the benchmarks in machine learning tasks but even though they post the best results, other lighter methods exist as well which have been used for a long time.

Deep learning is being used for different use cases in a multitude of areas, one of them being medical data.
Text classification is a task in automatic text processing that has many potential uses in the medical world.
For example, a model for classifying medical reports could be used to automatically define a proper diagnosis code for a new piece of text or to find older mislabelled texts.

The motivation for this thesis was to, firstly, define the feasibility and current state of deep learning for automatic text classification of medical text and secondly, to build a set of tools for quick and easy classifier training of a number of different models that the hospital can use.

This thesis gives an introduction to text classification, describes some of the most influential classifiers from the past and the present day, outlines the current state of machine learning -based text classification, and presents the most influential model architectures and training methods for them.
The medical report classification problem, available data, compute resources, chosen methods and results for the trained classifiers are presented and discussed at the latter part of the thesis, after which the final chapter concludes the thesis.
