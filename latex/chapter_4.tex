\chapter{Medical report document classification} \label{Medical report document classification}

\section{Problem definition} \label{Problem definition}

Doctors diagnose multiple patients each day from which a medical report is written and stored in a database.
The report usually consists of a written statement regarding the patient and the visit, and an according diagnosis code.
As doctors are overworked and short on time, this step is usually done in a hurry which increases the chance of human error in both the written text and assigned diagnosis code.
Thus, automatically defining a proper diagnosis code based on the written text would reduce the chance of making an error and save some time on the doctors part.
Additionally, such methods could be used to analyze past data in order to validate the correctness of already given diagnosis labels.


To find out if it would be possible to automatically define the diagnosis of texts to a high enough certainty, a couple of deep learning -based models were trained on a binary classification task where the model was to predict whether a given text should have a certain knee-related diagnosis code or not.
Since compute resources were a major constraint, only approaches with a reasonable single-gpu training time were considered.
Thus, a pretrained and finetuned ULMFiT and ELECTRA, and a finetuned FinBERT (pretrained by Virtanen et al.~\cite{virtanen2019}), were chosen for comparison.
In addition to comparing the different models with each other, each model --- excluding FinBERT, which provided a vocabulary of it's own from pretraining --- was trained multiple times with different sizes of a vocabulary which was extracted from the medical texts to see if vocabulary size meaningfully impacted the results.
Given the success of BERT and it's WordPiece tokenization in various NLP tasks, SentencePiece was chosen as a tokenizer for ULMFiT as well.

\section{Data} \label{Data}
\subsection{Medical reports}\label{Medical reports}
The medical data for the work was provided by Auria services at the Turku university hospital.
It consists of doctor's statements with corresponding diagnosis codes and other metadata, such as date of visit and hospital ward, of patients in the Turku-region.
No metadata was utilized in the training data, only the written text and the corresponding diagnosis code.
Since the actual diagnosis code for the text appeared often in it, all diagnosis codes in text were masked with a [CODE] token in the training data to prevent the model from learning from them.
HTML-tags such as <br> were removed in preprocessing and texts that were deemed too short (less than 80 characters) were also removed from the training set.

After extracting the documents that had one of two knee-related diagnosis codes as positive samples, the class balance for the dataset was 85\% negative and 15\% positive samples.
The dataset was eventually balanced by upsampling the positive samples.
Overall, the data consisted of 175000 documents with varying lengths.
Token amounts were dependent on the size of the vocabulary used.

\subsection{General Finnish}\label{General Finnish}
Since receiving access to the actual medical data took some time and due to the fact that it was not allowed to be taken out of the hospital, the chosen models and code were additionally trained and tested on a general Finnish corpus that Virtanen et al. compiled for FinBERT \cite{virtanen2019}.
It is composed from three different sources: news articles, online discussions, and documents crawled from the Finnish internet, and consists of 3.3 billion tokens from 234 million sentences.
The total size of the corpus is roughly 30 times the size of the Finnish Wikipedia.
The corpus was extensively preprocessed by filtering out documents that had too high a ratio of digits, uppercase or non-Finnish alphabetic characters, or low average sentence length.
Additionally, documents that featured 25\% or more duplication were removed as well as heuristically defined undesirables~\cite{virtanen2019}.


\section{Compute resources} \label{Compute resources}
Compute resources for the project were provided by Auria services for the medical models, and CSC for the general Finnish models.

\subsection{CSC} \label{CSC}
CSC (IT Center for Science Ltd.) is a non-profit state enterprise owned by the Finnish state and higher education institutions in Finland.
It offers compute resources for scientific purposes to universities and upkeeps the FUNET network, which is the Finnish national research and education network.
CSC operates two supercomputers, namely Taito and Puhti, and is working on a new supercomputer, Mahti, that is scheduled to open for use some time in 2020.
For this project, Puhti was chosen since it provides an ``artificial intelligence partition'' with access to GPU nodes with multiple Nvidia V100 graphics cards.

\subsubsection{Puhti}\label{Puhti}
Puhti was launched on September 2, 2019.
It is an Atos cluster system and has a variety of different node types.

Puhti has 682 CPU nodes, with a theoretical peak performance of 1,8 petaflops, and an AI partition of 80 GPU nodes with a peak performance of 2,7 petaflops.
Each node is equipped with two Intel Xeon processors, code name Cascade Lake, with 20 cores each running at 2,1 GHz.
Each GPU node also has four Nvidia Volta V100 GPUs with 32 GB of memory each.
The nodes are equipped with 384 GB of main memory and 3,6 TB of fast local storage.
The AI partition is engineered to allow GPU-intensive workloads to scale well across multiple nodes~\cite{zotero-178}.

When working with Puhti on this project, the workflow consisted of coding and testing the neural networks locally first, and then using Slurm to run batch jobs on Puhti.
This lead to some additional overhead in time for the project since working simultaneously on two environments proved quite arduous.
Additionally, keeping tabs on the versions of code was very important since it could be altered in both locations, thus git \cite{zotero-186} was used for this version control.

\subsubsection{Slurm}\label{Slurm}
Puhti uses the Slurm workload manager \cite{zotero-174} to handle scheduling jobs for compute clusters.
It is an open-source, fault-tolerant, and highly scalable cluster management and job scheduling system for Linux clusters.
First, it manages the allocation of exclusive and/or non-exclusive access to compute nodes to users for some duration of time during which they can perform work.
Second, it provides a framework for starting, executing, and monitoring work on the set of allocated nodes.
Finally, it manages a queue of pending work to arbitrate the contention for resources~\cite{zotero-176}.


\subsection{Turku University Hospital} \label{Turku University Hospital}
For training models on the clinical data, access was granted to \textbf{Blackbird}, a computer for artificial intelligence owned by Auria services, with four Nvidia V100 graphics cards allowing simultaneous training of multiple models.
All training and handling of clinical data happened on this computer which is located in the hospital and behind the hospital firewall in order to ensure that the sensitive data and resulting models did not accidentally or otherwise leak into the outer world.


Although the architecture could have managed training a medium-sized BERT, it was considered too long of a task to reserve the compute resources for.
Additionally, there wasn't enough domain-specific data for such a task.


The workflow on Blackbird consisted of connecting to the linux-based computer using secure shell (SSH), using screen~\cite{zotero-180} to multiplex the connection to multiple shells, and running a training process on each shell.
Jupyter notebooks~\cite{zotero-182} were also used on the machine for data visualization and prototyping, and were accessed locally by using ssh tunneling~\cite{zotero-184}.

\section{Methods}\label{Methods}
\subsubsection{ULMFiT}\label{ULMFiT}
The fastai v1 -library was used for implementing ULMFiT\footnote{https://github.com/fastai/fastai}.
As the library's support for SentencePiece\footnote{https://github.com/google/sentencepiece} was at the time quite limited, a considerable amount of custom code had to be written to incorporate the sub-word tokenizer in the training process.
The scripts used for training ULMFiT with SentencePiece are open-sourced and can be found from\footnote{https://github.com/invisiblesheep/ulmfit-sentencepiece}.

\subsubsection{ELECTRA}\label{ELECTRA}
For training ELECTRA, the pretraining and finetuning scripts were used from the official github repository\footnote{https://github.com/google-research/electra}.
The code was forked in order to make some changes to it regarding finetuning and evaluation parameters\footnote{https://github.com/invisiblesheep/electra}.

\subsubsection{BERT}\label{BERT}
For finetuning FinBERT, HuggingFace's transformers library was ultimately used \cite{wolf2020}.
It provides pretraining and finetuning scripts for a multitude of transformer-based models, such as BERT, XLNet and RoBERTa.
% The code for finetuning FinBERT was taken from the official BERT github repository\footnote{https://github.com/google-research/bert}, and FinBERT itself can be found from\footnote{https://github.com/TurkuNLP/FinBERT}.

\subsubsection{Baseline models}\label{Baseline models}
\textbf{fastText}\footnote{https://github.com/facebookresearch/fastText} was trained as a baseline model.
In addition to providing word embeddings, fastText can be used as a classifier as well.
fastText obtains document vectors by averaging word embeddings after which it uses multinomial logistic regression for classification.
As with most neural network classifiers, a probability distribution over classes is gained as a result after applying the softmax function to the results.
It uses a bunch of tricks, such as hierarchical softmax, to up the speed of training the model.
Thus it's an order of magnitude faster to train than a deep learning model but it still is somewhat competitive with one.

In addition to fastText, the results of a random classifier averaged over four runs and a majority predictor, which always predicts negative in this case, are also reported as a baseline.

\section{Results}\label{Results}
A binary classifier was built to identify texts that had one of two knee-related diagnosis codes from the training data.
The data was divided into training, validation and test sets with a 80--10--10 split, and stratified so that each set had an equal percentage wise representation of the classes.

\begin{table}[t]
\begin{center}
\begin{tabular}{lcccc}
         & Accuracy & Precision & Recall & F1 \\
        \hline
FinBERT &   91.68 &   68.02   &    74.16 &  70.96   \\
ELECTRA-30K    &  90.40 &   66.06 & 73.72 & 69.68 \\
ELECTRA-50K  &   91.23 &   69.29 & 76.19 & 72.58 \\
ELECTRA-100K  &  91.46 &   69.64 & 76.14 & 72.74 \\
ULMFiT-30K  &  93.85 &  84.28 & 72.54 & 77.97 \\
ULMFiT-50K  & 93.98 &  82.91 & 75.41 & 78.98 \\
ULMFiT-100K  & 94.11 & 83.95 & 75.17 & \textbf{79.32} \\
fastText &  93.49 & 84.34 & 70.07 & 76.54 \\
Random &  50.02 & 15.01 & 50.19 & 23.11 \\
Most common & 85.03 & N/A & 0 & N/A \\
\end{tabular}
\caption{Classification results of models with different vocabulary sizes on evaluation set.}
\label{table:results}
\end{center}
\end{table}

fastText's results were achieved by first running it's hyperparameter optimization command to find well-performing parameters after which the final parameters were forked manually.
The final hyperparameters were a learning rate of 0.05, 25 epochs of training and the usage of word bigrams.

Mainly default hyperparameters were used while training ULMFiT, except the learning rate was found with fast.ai's learning rate finder.
The underlying AWD-LSTM -architecture was not changed in any way.
ULMFiT was pretrained and finetuned for 5 and 12 epochs, respectively.
The finetuning results stopped improving after 8 finetuning epochs.

For finetuning ELECTRA, the best and final results were obtained with a maximum sequence length of 256, a learning rate of $10^{-4}$ and 12 epochs of training.
For FinBERT, final hyperparameters were a maximum sequence length of 256, learning rate of $2e^{-5}$ and 10 epochs.

This sequence length parameter is particular to the transformer-model.
It defines the maximum amount of tokens that can be input into the network at one time.
Given that the input documents for classification were oftentimes quite a bit longer than this imposed restriction, documents were chopped to multiple chunks with this maximum token amount to get around it.

\subsection{Tools}\label{Tools}
For training these models, a number of scripts were written and documented for the hospital to use.
The scripts comprise of a combination of shell and python scripts.
A general preprocessing script processes and splits the data of a specific given format of medical text data into a training and evaluation set.
Separate scripts are provided for pretraining and finetuning the various models.


\section{Discussion}\label{Discussion}
As seen from the results, the transformer-based models --- ELECTRA and FinBERT --- performed the worst of all the models.
These results should be taken with a grain of salt since the dataset wasn't quite large enough to fully utilize the power of these models.
It is suspected that a major factor for this lack of performance is due to the lack of a proper general medical text dataset that could be utilized in the pretraining of such models.
Additionally, since these transformers have a maximum sequence length for input tokens which prohibits the utilization of all the tokens in longer texts, the lengthier documents --- which there were a lot of --- in the training set could not be fully utilized.
Thus for FinBERT and ELECTRA, a major improvement came from increasing the maximum sequence length.


An additional factor for FinBERT's poor performance could be that the vocabulary it uses is not specific to the finetuning task as it was trained on general Finnish.
As such it does not include any of the medical text -related terms suspected to be important for classification, as opposed to the vocabularies of the other methods which were extracted from the training data.


Given that the relevant text regarding the diagnosis of the document seemed to usually reside at the end of a document, encapsulating this information in the training samples for ELECTRA and BERT perhaps needs a different approach than chunking to maximum size in which the relevant information only appears in the later chunks of a document.
This means that the previous chunks in a document can possibly negatively impact the performance of the classifier as well by making it focus on unimportant details.


The superior results of ULMFiT can also be due to the compatibility of LSTM-networks with long texts.
Contrary to the transformers which get as input the whole document, LSTM's are recurrently fed tokens one after the other until the whole document has gone through the network.
Thus, LSTM's have no limit in how long of a text it can process which proved to be an important feature with this dataset.
Additionally, as the important features of text were usually at the end of a document, the LSTM saw these features last and thus they had a relatively bigger impact on the network's parameters.


Class imbalance in the training data was suspected to be a factor for model performance, thus positive examples were upsampled by random duplication to balance the classes which overall somewhat improved the performance of all the models.
Comparing the different vocabulary sizes, the results clearly show that a bigger vocabulary size worked better than a smaller one on this dataset, which could be due to the fact that the data --- being medical text --- included a lot of medical terms rarely found in everyday language.

For future work, pretraining ELECTRA or BERT on a larger, general medical text corpus before finetuning on domain data could yield overall improvements to diagnosis code classifiers.
Pretraining on a larger dataset would give the models a good starting point for finetuning due to the fact that the models would then have a proper representation of the medical textual data domain, thus increasing the likelihood that the models generalize better.
Additionally, comparing the performance of the models with different amounts of training data would give insight into the limits of ULMFiT and ELECTRA.
