\documentclass[a4paper,12pt,language=finnish,version=draft,hidechapters=true,includereferences=false,realtimesnewroman=false,sharelatex=false,emptyfirstpages=true]{utuftthesis}
\setcounter{secnumdepth}{2}
\setcounter{tocdepth}{2}

\addbibresource{Bibliografia.bib}
\begin{document}
\begin{comment}
Document template suitable for use as a LaTeX master-file for master's
thesis in University of Turku Department of Future Technologies.\\
\\
Compatible with: ShareLaTeX / PDFLaTeX / XeLaTeX.\\
\\
\\
{*}{*} HOW TO USE? {*}{*}\\
\\
Want to write a thesis? Clone this template in ShareLaTeX or fork
the department's thesis git project.\\
\\
The utuftthesis.cls defines a new thesis class, which is based on
the report class. It supports these new named parameters:

- paper: a4paper

- version: draft / final (default: draft) shows/hides {[}draft{]}
in the header

- language: finnish / english (default: finnish) affects the general
document appearance and hyphenation

- hidechapters: true / false (default: true) hides/shows the chapter/luku
text at the beginning of each Chapter

- includereferences: true / false (default: false) include reference
pages when calculating the total number of pages

- realtimesnewroman: true / false (default: false) use Times New Roman
instead of LaTeX fonts with XeLaTeX. Requires the font to be installed
on the system / provided in the document directory. Other fonts can
be defined with \textbackslash setmainfont.

- sharelatex: true / false (default: false) don't attempt to use (c)
system fonts, instead read them from the project repository

- emptyfirstpages: true / false (default: true) clear the headers/footers
for the 1st pages of text chapters

Traditionally the best places to learn (La)TeX are probably the manual
pages for each package http://www.ctan.org/ and http://www.ctan.org/tex-archive/info/lshort/english/lshort.pdf.
This new version (2.0) should be compatible with xelatex and biblatex
which means that all source files can freely use normal UTF-8 text
without resorting to \textquotedbl\textquotedbl legacy hacks\textquotedbl .\\
\\
Note that PDF/A requirements don't allow PDF links, but if you want
to provide a user friendly version of the thesis with links, use \textbackslash hyperref\\
\\
\\
{*}{*} Maintenance {*}{*}\\
\\
Workflow: https://gitlab.utu.fi/ttweb/thesis -> master .lyx document
exported as .tex documents -> repository content dumped to the sharelatex
project template

Want to fix something in the template? Send a merge.\\
\\
Relies on utuftthesis.cls for the document class definitions.
\end{comment}


\pubyear{2018}

\pubmonth{6}

\publab{Labran nimi}

\publaben{Laboratory Name}

\pubtype{tkk}
\title{Name of Thesis}
\author{My Name}

\maketitle
\keywords{syväoppiminen, tekstinluokittelu, lääketieteellinen data}
\begin{abstract}
Tekstipohjaista tietoa tuotetaan vuosi vuodelta enemmän mikä puolestaan on lisännyt tarvetta automaattiselle tekstinkäsittelylle.
Täten myös automaattisia tekniikoita luonnollisen kielen käsittelyyn on enenevissä määrin tutkittu ja kehitetty, erityisesti viimeisen vuosikymmenen aikana.
Tämä on johtanut huomattaviin parannuksiin erilaisissa luonnollisen kielen käsittelytehtävissä.
Suuri läpimurto on ollut valtavilla tietomäärillä koulutettujen syvien neuroverkkojen käyttäminen.
Tällaisten menetelmien käyttö alueilla joilla aika on arvokasta, kuten lääketiede, voisi tarjota huomattavaa lisäarvoa.

Tämä tutkielma antaa yleiskuvan luonnollisen kielen käsittelystä keskittyen syväoppimiseen ja tekstinluokitteluun.
Lisäksi erilaisten syväoppivien menetelmien käytettävyyttä arvioitiin kouluttamalla tekstiluokittelijoita ennustamaan suomenkielisten lääketieteellisten dokumenttien diagnoosikoodeja.
Valittuihin menetelmiin kuuluvat syväoppimiseen perustuvat FinBERT, ULMFiT ja ELECTRA, sekä yksinkertaisempi lineaarinen luokittelija fastText.

Tulokset osoittavat, että rajallisella aineistolla lineaariset menetelmät, kuten fastText, toimivat yllättävän hyvin.
Syväoppimiselle perustuvat menetelmät taasen vaikuttavat toimivan kohtuullisen hyvin, vaikkakin niiden aito potentiaali pitäisi todentaa käyttäen suurempia datajoukkoja.
Täten jatkotutkimusta syväoppiviin menetelmiin liittyen tarvitaan.
\end{abstract}

\keywordstwo{deep learning, text classification, medical data }
\begin{abstracten}
Text-based data is produced at an ever growing rate each year which has in turn increased the need for automatic text processing.
Thus, to keep up with the amount of data, automatic natural language processing techniques have also been increasingly researched and developed, especially in the last decade or so.
This has lead to substantial improvements in various natural language processing tasks such as classification, translation and information retrieval.
A major breakthrough has been the utilization of deep neural networks and massive amounts of data to train them.
Using such methods in areas where time is valuable, such as the medical field, could provide considerable value.

In this thesis, an overview is given of natural language processing w.r.t deep learning and text classification.
Additionally, a dataset of medical reports in Finnish was preprocessed and used to train and evaluate a number of text classifiers for diagnosis code prediction in order to define the feasibility of such methods for medical text classification.
The chosen methods include deep learning -based FinBERT, ULMFiT and ELECTRA, and a simpler linear baseline classifier, fastText.

The results show that with a limited dataset, linear methods like fastText work surprisingly well.
Deep learning -based methods, on the other hand, seem work reasonably well, and show a lot of potential especially in utilizing larger amounts of training data.
In order to define the full potential of such methods, further investigation is required with different datasets and classification tasks.
\end{abstracten}


% empty pagestyle for table of contents etc.
% otherwise you'll get simple page style with roman page numbers
\pagestyle{empty}

% mandatory
\tableofcontents

% if you want a list of figures
%\listoffigures

% if you want a list of tables
%\listoftables

% 'list of acronyms'
%   - you may not need this at all
%   - create a chapter called List Of Acronyms (or whatever), which
%     should contain all your acronym definitions, e.g. 
%     \chapter{List Of Acronyms} 
%   - the secnumdepth trickery is needed because acronyms are as a
%     standard chapter and we are faking '\listofacronyms'
%
%\setcounter{secnumdepth}{-1}
%\input{your acronym chapter's file name}
%\setcounter{secnumdepth}{2}% setup page numbering, page counter, etc.%
\begin{comment}
The thesis starts here.

To better organize things, create a new tex file for each chapter
and input it below.

Avoid using the å, ä, ö or <space> characters in referred names and
underscores \_ in file names (may break hyperref).

Good luck!
\end{comment}

\input{johdanto.tex}\input{toinenluku.tex}

%\input{file_name_of_chapter_x}
%\input{file_name_of_chapter_y}

\begin{comment}
The thesis main content ends here.
\end{comment}
% \printbibliography

\begin{comment}
Create your appendix chapters with command \textbackslash appchapter\{some
name\} instead of \textbackslash chapter\{some name\} for the automagic
page counting to work!
\end{comment}


\appchapter{Liitedokumentti}

Tässä esimerkki\pagebreak{}

kaksisivuisesta liitteestä.

\appchapter{Liitedokumentti 2}

Tässä esimerkki\pagebreak{}

toisesta kaksisivuisesta liitteestä.

\begin{comment}
main document ends here
\end{comment}

\end{document}
