\chapter{Conclusion} \label{Conclusion}
In this thesis, an overview is given of natural language processing w.r.t deep learning and text classification.
Additionally, a dataset of medical reports was preprocessed and used to train a number of machine learning models on a binary classification task in order to define the feasibility of deep learning methods for medical text classification.

It was found that the LSTM-based ULMFiT performed the best out of the chosen models, others being ELECTRA, FinBERT and fastText.
The performance of a simpler method proved to be surprisingly good when compared to the more complex and training intensive deep neural networks.
This is a promising finding due to the fact that smaller hospitals don't have the resources to train and use the weightier deep neural networks in analyzing their documents.
Utilizing simpler and lighter models such as fastText levels the playing field and brings the benefits of AI to more people.

The lacklustre results of the transformer-based models are advised to be taken with a grain of salt as these approaches are deemed to require further investigation in this domain.

As a results of this thesis, the scripts for training the aforementioned models were compiled and provided to the hospital as tools for preprocessing medical data and pretraining and finetuning various classifiers for different diagnosis codes.

For future work, it is suggested that a larger general corpus of medical text is gathered and used to pre-train a deep neural network first in order to improve performance of the various introduced methods such as transformer-based BERT and ELECTRA.
